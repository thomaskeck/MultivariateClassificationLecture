\PassOptionsToPackage{usenames,dvipsnames,svgnames,table}{xcolor}
\documentclass[compress]{beamer}
\usetheme{CSC}

\usepackage{minted}
\newminted{cpp}{gobble=4,linenos}

\usepackage{graphics}
\graphicspath{{images/}}

\usepackage{pgf}
\usepackage{pgffor}
\usepackage{tikz}
\usepackage{pgfplots}
\usepgfplotslibrary{colormaps,fillbetween}
%\pgfplotsset{compat=1.14}
\usetikzlibrary{pgfplots.colorbrewer,pgfplots.fillbetween}

\usetikzlibrary{shadows.blur}
\usetikzlibrary{shapes.symbols}
\usetikzlibrary{calc,intersections,matrix}

\usepackage{ifthen}
\usetikzlibrary{arrows,fadings,automata,snakes,shapes,shapes.misc,shapes.arrows,trees,positioning,calc,decorations.pathreplacing,arrows.meta}
\usepackage[binary-units=true]{siunitx}
\usepackage{wasysym}
\usepackage{pgfplots}
\usepackage{listings}
\usepackage{xcolor}

\usepackage{tcolorbox}
\usepackage{booktabs}

\usepackage[framemethod=TikZ]{mdframed}
\mdfdefinestyle{simplebox}{roundcorner=4pt,linewidth=0,backgroundcolor=blue!50!black,fontcolor=white}

\usepackage{multicol}
\usepackage{etoolbox}

\makeatletter
\patchcmd{\beamer@sectionintoc}{\vskip1.5em}{\vskip1.0em}{}{}
\makeatother

\AtBeginSection[] {
  \begin{frame}
    \frametitle{\insertsection}
  \vspace{-0.5cm}
  \begin{center}
  \begin{scriptsize}
  \begin{multicols}{2}
    \tableofcontents[sectionstyle=show/shaded,subsectionstyle=show/show/hide]
    \end{multicols}
  \end{scriptsize}
  \end{center}
  \end{frame}
}


\tikzfading[name=arrowfading, top color=transparent!0, bottom color=transparent!95]
\tikzset{arrowfill/.style={top color=blue!50, bottom color=blue,}}
\tikzset{arrowstyle/.style={draw=black,arrowfill, single arrow,minimum height=#1, single arrow,
single arrow head extend=.4cm,}}

\tikzset{
    box/.style={
      rectangle,
	  color=#1,
      draw=black,
      fill=#1,
      thick,
      text=black,
      align=center,
      rounded corners=6pt,
      blur shadow={shadow blur steps=5},
      minimum height=1.5em
    }, 
    hbox/.style={
      rectangle,
      draw=black,
      fill=black,
      thick,
      text=white,
      align=center,
      rounded corners=6pt,
      blur shadow={shadow blur steps=5},
      minimum height=1.5em
    }, 
}

\colorlet{tree0}{blue}
\colorlet{tree100}{white!20!blue}
\colorlet{tree200}{white!40!blue}
\colorlet{tree300}{white!60!blue}
\colorlet{tree400}{white!80!blue}
\colorlet{tree500}{white}
\colorlet{tree600}{white!80!orange}
\colorlet{tree700}{white!60!orange}
\colorlet{tree800}{white!40!orange}
\colorlet{tree900}{white!20!orange}
\colorlet{tree1000}{orange}
